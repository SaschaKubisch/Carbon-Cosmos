\documentclass{article}
\usepackage[utf8]{inputenc}
\usepackage{listings}
\usepackage{minted}
\usepackage{hyperref}
\usepackage{tocloft}
\lstset{
language=Go,
basicstyle=\ttfamily\small,
frame=single,
showstringspaces=false,
commentstyle=\color{gray},
keywordstyle=\color{blue},
}


\title{Table of Contents}
\author{}
\date{}

\begin{document}

\maketitle

\tableofcontents

\section{Introduction to Golang}

Golang, or Go, is an open-source programming language developed by Google. It is designed with simplicity, efficiency, and concurrency in mind. Go is statically typed and compiled, making it ideal for systems programming, backend development, and other performance-critical applications. This chapter provides an introduction to the basic syntax of Golang.

\subsection{Basics of Golang Syntax}

\subsubsection{Variables and Data Types}

In Go, variables are declared using the \texttt{var} keyword, followed by the variable name and data type. Variables can be initialized with a value when they are declared.

\begin{minted}{go}
var x int
var y int = 10
\end{minted}

Go also allows short variable declarations using the \texttt{:=} operator. The data type is inferred from the value.

\begin{minted}{go}
z := 20
\end{minted}

Some common data types in Go are \texttt{int}, \texttt{float64}, \texttt{bool}, and \texttt{string}.

\subsubsection{Control Structures}

Go supports \texttt{if}, \texttt{else if}, and \texttt{else} for conditional branching.

\begin{minted}{go}
if x > y {
fmt.Println("x is greater than y")
} else if x < y {
fmt.Println("x is less than y")
} else {
fmt.Println("x is equal to y")
}
\end{minted}

For loops, Go provides the \texttt{for} keyword.

\begin{minted}{go}
for i := 0; i < 10; i++ {
fmt.Println(i)
}
\end{minted}

\subsubsection{Functions}

Functions in Go are defined using the \texttt{func} keyword, followed by the function name, a list of input parameters, and a list of return types. Functions can return multiple values.

\begin{minted}{go}
func add(x int, y int) int {
return x + y
}

func swap(x int, y int) (int, int) {
return y, x
}
\end{minted}

\subsubsection{Packages and Imports}

Go organizes code into packages. To use a package, you need to import it using the \texttt{import} keyword.

\begin{minted}{go}
package main

import (
"fmt"
)

func main() {
fmt.Println("Hello, World!")
}
\end{minted}

This covers the basics of Golang syntax. In the following chapters, we will dive deeper into more advanced Golang concepts and how to use the language for smart contract development in the Cosmos ecosystem.


\subsection{Golang data types and control structures}


In this section, we will discuss the various data types available in Go and the control structures that help manage the flow of a program.

\subsubsection{Data Types}

Golang has several built-in data types. They can be categorized as follows:

\begin{itemize}
\item \textbf{Numeric types:}
\begin{itemize}
\item Integer types: int8, int16, int32, int64, uint8, uint16, uint32, uint64, int, uint, and uintptr
\item Floating-point types: float32 and float64
\item Complex number types: complex64 and complex128
\end{itemize}
\item \textbf{String type:} string
\item \textbf{Boolean type:} bool
\item \textbf{Derived types:}
\begin{itemize}
\item Pointer types: e.g., *int, *string
\item Array types: e.g., [3]int, [5]string
\item Slice types: e.g., []int, []string
\item Struct types: e.g., struct {Name string; Age int}
\item Interface types: e.g., interface {Read(p []byte) (n int, err error)}
\item Map types: e.g., map[string]int, map[int]string
\item Channel types: e.g., chan int, chan string
\end{itemize}
\end{itemize}

\subsubsection{Control Structures}

Control structures in Go help manage the flow of execution in a program. They include:

\begin{itemize}
\item \textbf{If-else statements:} Conditionally execute a block of code based on a boolean expression.
\begin{minted}{go}
if x > 10 {
    fmt.Println("x is greater than 10")
} else if x == 10 {
    fmt.Println("x is equal to 10")
} else {
    fmt.Println("x is less than 10")
}
\end{minted}

\item \textbf{For loops:} Repeat a block of code for a specified number of iterations, or until a condition is met.

\begin{minted}{go}
// For loop with a condition
for i := 0; i < 10; i++ {
    fmt.Println(i)
}

// For loop with a range
nums := []int{2, 4, 6, 8}
for index, value := range nums {
    fmt.Printf("Index: %d, Value: %d\n", index, value)
}
\end{minted}

\item \textbf{Switch statements:} Execute a block of code based on the value of an expression, with multiple cases and a default case.

\begin{minted}{go}
switch grade {
case "A":
    fmt.Println("Excellent!")
case "B":
    fmt.Println("Well done!")
case "C":
    fmt.Println("Good job!")
case "D":
    fmt.Println("You passed.")
case "F":
    fmt.Println("Better luck next time.")
default:
    fmt.Println("Invalid grade.")
}
\end{minted}
\end{itemize}

These data types and control structures form the foundation of Go programming. In the next sections, we will explore more advanced topics, such as functions, interfaces, error handling, and concurrency in Golang.


\subsection{Functions and interfaces in Golang}
\subsubsection{Functions}

Functions are the building blocks of a Go program. They allow you to define reusable pieces of code that can be called with arguments and return values. Functions are declared with the func keyword, followed by the function name, parameter list, return type, and function body.

\begin{minted}{go}
func functionName(parameter1 type1, parameter2 type2) returnType {
// function body
}
\end{minted}

Here's an example of a function that calculates the sum of two integers:

\begin{minted}{go}
func add(x int, y int) int {
return x + y
}

func main() {
result := add(10, 20)
fmt.Println(result)
}
\end{minted}

\subsubsection{Interfaces}

Interfaces are a powerful feature in Go that allows you to define a set of methods that a type must implement. They provide a way to achieve polymorphism and decouple code by specifying the behavior of objects rather than their concrete types.

An interface is defined using the type keyword followed by the interface name and the keyword interface. Inside the interface definition, you list the method signatures that a type must implement to satisfy the interface.

\begin{minted}{go}
type Reader interface {
Read(p []byte) (n int, err error)
}
\end{minted}

In this example, any type that implements a Read method with the specified signature is considered to satisfy the Reader interface. Implementing an interface is implicit in Go, which means you don't need to explicitly declare that a type implements an interface.





\subsection{Error handling and concurrency in Golang}

\subsubsection{Error Handling}

Error handling in Go is an essential aspect of writing robust and reliable code. Go uses a unique approach to error handling compared to other languages, where errors are treated as values and are returned explicitly by functions. The standard library provides an error interface, which is the conventional type for error values.

\begin{minted}{go}
type error interface {
Error() string
}
\end{minted}

To handle errors, you can return an error value as the last return value of a function. If the error value is not nil, it indicates that an error has occurred.

\begin{minted}{go}
func divide(x, y float64) (float64, error) {
if y == 0 {
return 0, errors.New("division by zero")
}
return x / y, nil
}

func main() {
result, err := divide(10, 0)
if err != nil {
fmt.Println("Error:", err)
} else {
fmt.Println("Result:", result)
}
}
\end{minted}

\subsubsection{Concurrency}

Go provides built-in support for concurrency using goroutines and channels. Concurrency allows you to run multiple tasks concurrently, which can lead to more efficient and responsive programs.

\begin{itemize}
\item \textbf{Goroutines:} A goroutine is a lightweight thread of execution managed by the Go runtime. To start a goroutine, simply use the go keyword followed by a function call.
\begin{minted}{go}
func printNumbers() {
    for i := 1; i <= 10; i++ {
        fmt.Println(i)
        time.Sleep(1 * time.Second)
    }
}

func main() {
    go printNumbers() // start a goroutine
    time.Sleep(5 * time.Second)
    fmt.Println("Main function finished.")
}
\end{minted}
\item \textbf{Channels:} Channels are a synchronization mechanism that allows you to send and receive values between goroutines. They provide a way to coordinate the execution of goroutines and to share data safely.

To create a channel, use the `make` function with the `chan` keyword followed by the type of values the channel will carry:

\begin{minted}{go}
messages := make(chan string)
\end{minted}

To send a value to a channel, use the channel's `<-` operator:

\begin{minted}{go}
messages <- "Hello, World!"
\end{minted}

To receive a value from a channel, use the channel's `<-` operator on the left-hand side of an assignment:

\begin{minted}{go}
message := <-messages
\end{minted}

Here's an example that demonstrates the use of goroutines and channels:

\begin{minted}{go}
func sendMessages(messages chan<- string) {
    for i := 1; i <= 5; i++ {
        messages <- fmt.Sprintf("Message %d", i)
        time.Sleep(1 * time.Second)
    }
    close(messages)
}

func main() {
    messages := make(chan string)

    go sendMessages(messages)

    for message := range messages {
        fmt.Println(message)
    }
}
\end{minted}
\end{itemize}

\subsection{Golang packages and modules}

\subsubsection{Packages}

Packages in Go are a way to organize and reuse code. A package is a collection of source files in the same directory that are compiled together. Functions, types, and variables defined in one source file are accessible to all other source files within the same package. To use code from another package, you need to import it using the import keyword.

A simple example of package usage:

\begin{minted}{go}
package main

import (
"fmt"
"math"
)

func main() {
fmt.Println("Square root of 4:", math.Sqrt(4))
}
\end{minted}

\subsubsection{Exported names}

In Go, a name is exported if it starts with a capital letter. Only exported names are accessible from other packages. In the example above, Sqrt is an exported function from the math package.

\subsubsection{Modules}

Go modules are a way to manage dependencies in your Go projects. A module is a collection of related Go packages that are versioned together as a single unit. Modules record precise dependency requirements and create reproducible builds.

To initialize a new module, use the go mod init command:

\begin{minted}{bash}
go mod init example.com/myproject
\end{minted}

This command creates a go.mod file in the root of your project, which contains information about the module, its dependencies, and the Go version used.

To add a dependency to your module, simply import it in your code, and the Go tools will automatically update your go.mod file.

Here's an example of a simple go.mod file:

\begin{minted}{text}
module example.com/myproject

go 1.16

require github.com/some/dependency v1.2.3
\end{minted}

This covers the basics of Golang packages and modules. In the following chapters, we will explore the Cosmos SDK, learn about custom modules, and discuss tools for testing, deploying, and interacting with smart contracts in the Cosmos ecosystem.
%
%
\section{Overview of the Cosmos Ecosystem}

\subsection{Introduction to Cosmos SDK and Tendermint}

\subsubsection{Cosmos SDK}

The Cosmos SDK is a framework for building blockchain applications in Golang. It provides a set of tools and libraries to create custom, modular, and interoperable blockchain applications that leverage Tendermint, a high-performance and secure consensus engine.

Some key features of the Cosmos SDK include:

\begin{itemize}
\item Modular architecture: Developers can build and customize their blockchain applications using pre-built modules or create custom modules.
\item Capability-based security: Access control is enforced through object capabilities, which provides a secure and flexible system for managing permissions.
\item Performance and scalability: The Cosmos SDK is designed to optimize for high throughput, low latency, and low storage overhead.
\end{itemize}

\subsubsection{Tendermint}

Tendermint is a Byzantine Fault Tolerant (BFT) consensus engine that powers Proof of Stake (PoS) blockchains. It provides a secure and efficient solution for maintaining consensus across a distributed network of nodes. Tendermint is designed to be used with the Cosmos SDK but can also be used as a standalone consensus engine for other blockchain projects.

Some key features of Tendermint include:

\begin{itemize}
\item Instant finality: Tendermint guarantees that once a block is committed, it cannot be reverted or forked.
\item Light client support: Tendermint provides a lightweight client protocol that allows low-resource devices to securely interact with the blockchain.
\item Simple and modular: Tendermint has a clean and modular design that makes it easy to understand, use, and extend.
\end{itemize}

Together, the Cosmos SDK and Tendermint form the core of the Cosmos ecosystem, enabling developers to build scalable and interoperable blockchain applications with ease. In the following chapters, we will explore the concepts of Cosmos Hub and Zones, the Inter-Blockchain Communication (IBC) protocol, and compare the Ethereum and Cosmos ecosystems.

\subsection{Understanding the Cosmos Hub and Zones}

The Cosmos ecosystem is designed to enable the creation of interconnected, interoperable blockchains that can communicate with each other seamlessly. To achieve this, the Cosmos architecture is divided into two main components: the Cosmos Hub and Zones.

\subsubsection{Cosmos Hub}

The Cosmos Hub is the central blockchain in the Cosmos ecosystem, built using the Cosmos SDK and powered by the Tendermint consensus engine. It is responsible for connecting and maintaining the security of all connected blockchains, or Zones. The Cosmos Hub manages the interchain communication through the Inter-Blockchain Communication (IBC) protocol and ensures that cross-chain transactions are securely executed.

The native token of the Cosmos Hub is the Atom, which is used for staking, governance, and to pay transaction fees. Validators, who are responsible for securing the network and validating transactions, are selected based on the amount of Atoms they have staked.

\subsubsection{Zones}

Zones are individual blockchains that are connected to the Cosmos Hub. Each Zone is an independent blockchain built using the Cosmos SDK and has its own validator set, consensus mechanism, and state machine. Zones can have their own native tokens and can implement custom functionality to cater to specific use cases.

Zones communicate with the Cosmos Hub and other Zones using the IBC protocol, allowing them to exchange tokens and data in a trustless and decentralized manner. This enables the creation of a scalable and interoperable ecosystem of blockchains that can work together to solve complex problems.

In summary, the Cosmos Hub and Zones form the foundation of the Cosmos ecosystem, enabling the creation of scalable and interoperable blockchain applications. The Cosmos Hub acts as the central connecting point for all Zones, while Zones are independent blockchains with their own consensus mechanisms and state machines. Together, they facilitate the seamless exchange of tokens and data across the entire Cosmos ecosystem.
%
%
\subsection{Inter-Blockchain Communication (IBC) protocol}
%
The Inter-Blockchain Communication (IBC) protocol is a core innovation in the Cosmos ecosystem that enables seamless communication between independent blockchains, called Zones. IBC facilitates the transfer of tokens, data, and other assets across different blockchains while maintaining the security and sovereignty of each chain.
%
\subsubsection{IBC Architecture}
%
IBC is designed as a layered protocol with clear separation of concerns, making it modular and extensible. The main components of the IBC architecture are:
%
\begin{itemize}
\item \textbf{Clients}: Clients are light client implementations of other blockchains that run within a Zone. They are responsible for tracking the state and consensus updates of the connected chains, ensuring that the IBC protocol can verify proofs and maintain security.
\item \textbf{Connections}: Connections are the secure communication channels between two chains, established by mutual consent. They are built on top of the Clients and require both chains to trust each other's light client implementations.
\item \textbf{Channels}: Channels are higher-level abstractions built on top of Connections. They allow for the ordered or unordered exchange of packets between two chains and provide guarantees on packet delivery, timeouts, and acknowledgments.
\item \textbf{Packets}: Packets are the data payloads that are transmitted between chains via Channels. They can contain tokens, data, or any other custom information that needs to be transferred across chains.
\end{itemize}
%
\subsubsection{IBC Workflow}
%
The IBC protocol follows a specific workflow to establish secure communication between two chains:
%
\begin{enumerate}
\item Chains create and initialize their respective Clients, which track the state and consensus updates of the other chain.
\item The two chains establish a secure Connection using their Clients. This requires a handshake process in which both chains verify each other's light client implementations.
\item Once the Connection is established, the chains can create one or more Channels to facilitate the exchange of packets.
\item The chains can now send and receive packets using the established Channels. Packets are verified, processed, and acknowledged by both chains, ensuring secure and trustless communication.
\end{enumerate}
%
In conclusion, the Inter-Blockchain Communication (IBC) protocol is a key innovation in the Cosmos ecosystem that enables seamless and secure communication between independent blockchains. Its modular architecture, consisting of Clients, Connections, Channels, and Packets, allows for a flexible and extensible approach to cross-chain communication. By facilitating the exchange of tokens, data, and other assets across chains, IBC paves the way for a decentralized and interoperable ecosystem of blockchains.
%
\subsection{Comparison between Ethereum and Cosmos}
%
Ethereum and Cosmos are both popular platforms for building decentralized applications (dApps) and smart contracts. While they share some similarities, they have distinct differences in terms of their architecture, consensus mechanisms, and developer experience.
%
\subsubsection{Architecture}
%
\textbf{Ethereum:} Ethereum is a single blockchain with a global state that supports smart contracts written in the Solidity programming language. Ethereum's architecture is based on the Ethereum Virtual Machine (EVM), which executes smart contracts and maintains the global state.
%
\textbf{Cosmos:} Cosmos is a network of independent blockchains, called Zones, connected via the Inter-Blockchain Communication (IBC) protocol. Each Zone has its own state, consensus mechanism, and application logic. Cosmos allows developers to build custom blockchain applications using the Cosmos SDK, a modular framework built on top of the Tendermint consensus engine.
%
\subsubsection{Consensus Mechanism}
%
\textbf{Ethereum:} Ethereum currently uses Proof of Work (PoW) as its consensus mechanism. However, it is in the process of transitioning to Ethereum 2.0, which will implement a Proof of Stake (PoS) consensus mechanism called the Beacon Chain.
%
\textbf{Cosmos:} Cosmos uses a Byzantine Fault Tolerant (BFT) consensus algorithm called Tendermint. Tendermint is a PoS-based consensus mechanism that provides fast finality, high throughput, and strong safety guarantees.
%
\subsubsection{Developer Experience}
%
\textbf{Ethereum:} Ethereum developers typically use Solidity to write smart contracts and tools like Truffle, Hardhat, and Web3.js for development, testing, and deployment. Ethereum has a large and active developer community, with many resources available for learning and support.
%
\textbf{Cosmos:} Cosmos developers use the Cosmos SDK and the Golang programming language to build custom blockchain applications. The SDK provides a modular architecture that simplifies the development process, while IBC enables seamless interoperability between different Zones. Tools like Starport streamline the process of building, testing, and deploying Cosmos-based applications.
%
\subsubsection{Scalability and Interoperability}
%
\textbf{Ethereum:} Ethereum faces scalability challenges due to its global state and PoW consensus mechanism. Layer-2 solutions like Optimistic Rollups and ZK-Rollups are being developed to address these issues. Ethereum also has limited interoperability with other blockchains, typically relying on centralized or custodial bridges.
%
\textbf{Cosmos:} Cosmos addresses scalability by enabling the creation of multiple, independent Zones with their own consensus mechanisms and state. The IBC protocol allows these Zones to communicate and transfer assets seamlessly, providing a more scalable and interoperable ecosystem.
%
In summary, Ethereum and Cosmos offer distinct approaches to building decentralized applications, with Ethereum focusing on a single, global blockchain with smart contract support, and Cosmos enabling a network of interoperable blockchains with custom application logic.
%
%
\section{Setting Up the Development Environment}

In this section, we will discuss how to set up your development environment for working with Golang and the Cosmos ecosystem. We will cover the installation of Golang, Cosmos SDK, and related tools, as well as setting up IDEs and code editors for Golang development.

\subsection{Installing Golang, Cosmos SDK, and related tools}

\begin{enumerate}
\item \textbf{Install Golang:} Download and install the latest version of Golang from the official website: \url{https://golang.org/dl/}. Follow the installation instructions for your specific operating system.


\item \textbf{Install Cosmos SDK:} To install the Cosmos SDK, run the following command in your terminal or command prompt:

\begin{lstlisting}[language=bash]
go get -u github.com/cosmos/cosmos-sdk
\end{lstlisting}

This will download and install the Cosmos SDK in your \$GOPATH.

\item \textbf{Install Starport:} Starport is a rapid development tool for building Cosmos SDK applications. Install it by running the following command:

\begin{lstlisting}[language=bash]
curl -sSfL https://install.starport.network/starport! | sudo bash
\end{lstlisting}

This command downloads and installs Starport on your system.

\item \textbf{Install CosmWasm:} If you plan to work with CosmWasm smart contracts, follow the installation instructions in the CosmWasm documentation: \url{https://docs.cosmwasm.com/docs/getting-started/installation}.
\end{enumerate}

\subsection{Setting up IDEs and code editors for Golang development}

There are several IDEs and code editors that provide excellent support for Golang development. Some popular options include:

\begin{itemize}
\item \textbf{Visual Studio Code:} Install the Go extension by Microsoft from the Visual Studio Code marketplace. This extension provides features like code completion, linting, and debugging for Golang.


\item \textbf{GoLand:} GoLand is a dedicated Golang IDE developed by JetBrains. It offers robust support for Golang, including code completion, refactoring, and debugging. Download and install GoLand from the JetBrains website: \url{https://www.jetbrains.com/go/}.

\item \textbf{Sublime Text:} Sublime Text is a popular cross-platform code editor. Install the GoSublime package for Golang support, including syntax highlighting, code completion, and build integration.
\end{itemize}

After setting up your preferred IDE or code editor, ensure that your $GOPATH and $GOROOT environment variables are properly configured.

\subsection{Analogous tools to Hardhat, Web3.js, and Ganache}

In the Cosmos ecosystem, there are several tools that serve similar purposes to Hardhat, Web3.js, and Ganache in the Ethereum ecosystem:

\begin{itemize}
\item \textbf{Starport (Analogous to Hardhat):} As mentioned earlier, Starport is a rapid development tool for building Cosmos SDK applications. It streamlines the process of creating, testing, and deploying applications, similar to Hardhat in the Ethereum ecosystem.

css
Copy code
\item \textbf{Cosmos.js (Analogous to Web3.js):} Cosmos.js is a JavaScript library for interacting with Cosmos SDK applications. It provides functions for querying the blockchain state, constructing and signing transactions, and broadcasting transactions, similar to Web3.js for Ethereum.

\item \textbf{SimApp (Analogous to Ganache):} SimApp is a local in-memory blockchain simulator that is part of the Cosmos SDK. It allows developers to quickly test their applications and smart contracts without having to deploy them to an actual blockchain, similar to Ganache in the Ethereum ecosystem. SimApp is used in conjunction with the Cosmos SDK's simulation testing framework.
\end{itemize}

Now that you have your development environment set up and are familiar with the analogous tools in the Cosmos ecosystem, you can start building applications using Golang and the Cosmos SDK. The following sections will cover more advanced topics, such as creating custom modules, working with IBC, and developing smart contracts using CosmWasm.
%
%
\section{Cosmos SDK Basics}
%
\subsection{Application structure and components}
In this section, we will discuss the application structure and components of a typical Cosmos SDK blockchain application. Understanding the structure and components is essential for developing and modifying Cosmos SDK applications effectively.

A Cosmos SDK application is primarily composed of the following components:

\begin{enumerate}
\item \textbf{App:} The core of a Cosmos SDK application, responsible for managing the state, routing messages, and handling transactions. The app is defined in the \texttt{app.go} file in the application's root directory.

\item \textbf{Modules:} Modules are the building blocks of a Cosmos SDK application, providing specific functionalities. A module typically includes types, messages, handlers, and a keeper. Modules can be reused across different applications, and the Cosmos SDK comes with several built-in modules, such as \texttt{bank}, \texttt{staking}, and \texttt{governance}.

\item \textbf{Handlers:} Handlers process messages and update the application state based on the business logic of a module. Handlers are defined in the \texttt{handler.go} file within a module's directory.

\item \textbf{Keepers:} Keepers are responsible for managing the state of a module. They provide an interface to interact with the application state and perform CRUD operations on the module's data. Keepers are defined in the \texttt{keeper} subdirectory of a module's directory.

\item \textbf{Store:} The store is a key-value database that holds the application's state. Each module has its own store, which is accessible through the module's keeper.

\item \textbf{Messages:} Messages are used to trigger state transitions in a module. They are typically created by external clients, such as wallets or other applications, and are included in transactions that are submitted to the blockchain. Messages are defined in the \texttt{types/msgs.go} file within a module's directory.

\item \textbf{Transactions:} Transactions are the primary means of interacting with a Cosmos SDK application. They contain one or more messages, signatures from the message signers, and transaction fees.

\item \textbf{Events:} Events are emitted by the application when certain actions occur, such as a message being processed or a block being committed. Events can be used to track the state of the application or to trigger off-chain actions.
\end{enumerate}

The application structure of a Cosmos SDK project usually follows this pattern:

\begin{itemize}
\item \texttt{app/}
\begin{itemize}
\item \texttt{app.go} - Application definition and main entry point
\end{itemize}
\item \texttt{cmd/}
\begin{itemize}
\item \texttt{d/} - Daemon for the application
\item \texttt{cli/} - Command line interface for the application
\end{itemize}
\item \texttt{x/} - Custom modules
\begin{itemize}
\item \texttt{<module>/}
\begin{itemize}
\item \texttt{types/} - Types, messages, and events
\item \texttt{keeper/} - Keeper definition
\item \texttt{handler.go} - Handler definition
\item \texttt{module.go} - Module definition
\end{itemize}
\end{itemize}
\end{itemize}

In the following sections, we will delve deeper into these components and learn how to create custom modules for a Cosmos SDK application.
%

\subsection{Modules, keepers, and handlers}

In this section, we will provide a more in-depth explanation of modules, keepers, and handlers, which are the primary components of a Cosmos SDK application.

\textbf{Modules:} A module is a self-contained, reusable piece of code that provides specific functionality within a Cosmos SDK application. Modules are the building blocks of an application and can be combined to create a fully functional blockchain. A Cosmos SDK application is essentially a collection of modules that work together to manage the blockchain state and process transactions. Each module typically includes the following components:

\begin{itemize}
\item Types
\item Messages
\item Handler
\item Keeper
\end{itemize}

\textbf{Keepers:} Keepers are responsible for managing the state of a module. They provide an interface to interact with the application state and perform CRUD operations on the module's data. Each module has its own keeper, which is responsible for managing the module's state. Keepers are defined in the \texttt{keeper} subdirectory of a module's directory.

The keeper interacts with the store, a key-value database that holds the application state. The keeper provides methods for reading and writing data from the store, as well as performing more complex operations such as iterating through keys or performing queries. Keepers are also responsible for managing access control to the module's state, ensuring that only authorized components can perform specific operations.

\textbf{Handlers:} Handlers process messages and update the application state based on the business logic of a module. They are the entry point for processing transactions that contain messages related to a specific module. Handlers are responsible for validating the message, ensuring it adheres to the module's requirements, and performing any necessary state transitions. Handlers are defined in the \texttt{handler.go} file within a module's directory.

The handler receives a message and a reference to the application's context, which contains the current state of the application. The handler then uses the module's keeper to read or update the state as needed. Once the state has been updated, the handler returns a result that contains information about the transaction, such as the updated state, events, or any relevant data.

In summary, modules, keepers, and handlers are the primary components of a Cosmos SDK application. Modules provide specific functionality, keepers manage the state of a module, and handlers process messages and update the state according to the module's business logic. Understanding these components is essential for developing and modifying Cosmos SDK applications effectively.
\subsection{Store and state management in Cosmos SDK}

In the Cosmos SDK, the store is the key component responsible for managing the application state. The store is a key-value database that holds the state of the blockchain. In this section, we will discuss how the store works and its role in state management in Cosmos SDK applications.

\textbf{Store:} The Cosmos SDK uses a Merkle tree-based key-value store called the MultiStore, which consists of several sub-stores. Each sub-store is responsible for managing the state of a specific module. The MultiStore ensures that the state of each module is kept separate and isolated from the others. The state of the entire application is represented by the root hash of the Merkle tree, which is computed from the individual state roots of each sub-store.

\textbf{KVStore:} The primary sub-store type used in the Cosmos SDK is the KVStore (key-value store). The KVStore provides a simple interface for reading and writing key-value pairs. Each module's keeper interacts with the KVStore to perform CRUD operations on the module's state data. KVStores use a prefix-based system to partition the keyspace, ensuring that keys do not conflict between modules.

\textbf{State Commitment:} One of the key features of the Cosmos SDK is its state commitment mechanism, which allows for provable and verifiable state updates. When a block is committed, the application state is updated, and the new root hash of the Merkle tree is calculated. This root hash serves as a commitment to the entire state of the application and is included in the block header. Clients can use this root hash and Merkle proofs to verify the state of specific keys or values without needing to download the entire state.

\textbf{State Versioning:} The Cosmos SDK supports state versioning, allowing applications to maintain multiple versions of the state simultaneously. This is particularly useful for historical queries and fast syncing. When a new block is committed, the state is updated, and a new version is created. Older versions of the state can still be accessed and queried, but updates can only be performed on the latest version.

In conclusion, the store is the central component responsible for managing the state of a Cosmos SDK application. It uses a Merkle tree-based key-value store called the MultiStore, with separate sub-stores for each module's state. The store provides a provable and verifiable state commitment mechanism, as well as support for state versioning. Understanding the store and state management is crucial for developing and maintaining Cosmos SDK applications effectively.
\subsection{Messages, transactions, and events}

In the Cosmos SDK, messages, transactions, and events are essential components that facilitate interactions within the blockchain network. In this section, we will discuss these components and their roles in the Cosmos SDK.

\textbf{Messages:} Messages are the fundamental building blocks that represent actions within a Cosmos SDK application. A message is a data structure that defines a specific operation, such as transferring tokens, creating a new validator, or interacting with a smart contract. Each module defines its own set of messages and corresponding message handlers to process them.

\textbf{Transactions:} Transactions are the containers that group one or more messages to be executed atomically within a single block. A transaction also includes metadata, such as a fee, a gas limit, and a signature from the sender to ensure the authenticity and validity of the transaction. Transactions are submitted by users and relayed through the network until they are included in a block and executed by the validators.

\textbf{Transaction Processing:} When a transaction is executed, each message within the transaction is processed sequentially by its corresponding message handler. If any message fails to execute, the entire transaction is reverted, and any state changes are discarded. The transaction's sender is charged for the gas consumed during the execution, regardless of whether the transaction succeeds or fails.

\textbf{Events:} Events are emitted during the execution of messages to provide information about the state changes that have occurred. Events are typically used to notify external systems, such as wallets, block explorers, or other applications, about relevant occurrences within the blockchain. Events can be filtered and indexed, allowing subscribers to receive only the events that they are interested in.

In summary, messages, transactions, and events are essential components in the Cosmos SDK that enable users to interact with the blockchain network. Messages represent individual actions, transactions group messages to be executed atomically, and events provide information about state changes during message execution. Understanding these components is crucial for developing and interacting with Cosmos SDK applications effectively.

%
%
\section{Creating Custom Modules}
%
\subsection{Defining module types, messages, and events}

In the Cosmos SDK, modules are the building blocks that define the functionality of an application. To create a custom module, developers need to define its types, messages, and events. In this section, we will discuss how to define these components within a custom module.

\textbf{Module Types:} Each module has its own set of types, which are data structures that define the state of the module. For example, a module that manages NFTs would define a type to represent the NFT itself, including its owner, metadata, and other relevant properties. To define a module type, create a new file in the module's types folder and define a Golang struct that represents the desired data structure.

\textbf{Messages:} Messages are the primary means by which users interact with a module. Each message corresponds to a specific action, such as creating an NFT, transferring ownership, or updating metadata. To define a message, create a new Golang struct in the module's types folder and implement the sdk.Msg interface. This interface requires implementing methods like Route, Type, ValidateBasic, and GetSignBytes.

\textbf{Events:} Events are emitted by message handlers during the execution of messages to provide information about state changes that have occurred. Events can be used to notify external systems about relevant occurrences within the blockchain. To define an event, create a new Golang struct in the module's types folder and implement the sdk.Event interface. This interface requires implementing methods like Type, Attributes, and ToABCIEvent.

To illustrate these concepts, let's consider an example of creating a simple NFT module. First, we would define the NFT type within a file named nft.go in the types folder:

\begin{lstlisting}[language=Go]
package types

type NFT struct {
ID string
Owner sdk.AccAddress
Metadata string
}
\end{lstlisting}

Next, we would define the messages for creating and transferring NFTs within a file named msgs.go in the types folder:

\begin{lstlisting}[language=Go]
package types

type MsgCreateNFT struct {
Owner sdk.AccAddress
Metadata string
}

type MsgTransferNFT struct {
Sender sdk.AccAddress
Recipient sdk.AccAddress
NFTID string
}
\end{lstlisting}

Finally, we would define the events for NFT creation and transfer within a file named events.go in the types folder:

\begin{lstlisting}[language=Go]
package types

type EventCreateNFT struct {
Owner sdk.AccAddress
NFTID string
}

type EventTransferNFT struct {
Sender sdk.AccAddress
Recipient sdk.AccAddress
NFTID string
}
\end{lstlisting}

By defining the module types, messages, and events, developers can create custom modules that encapsulate the desired functionality and enable users to interact with the blockchain application effectively.
\subsection{Implementing handlers and keepers}

Handlers and keepers are essential components of a Cosmos SDK module, responsible for processing messages and managing the module's state. In this section, we will discuss how to implement handlers and keepers within a custom module.

\textbf{Handlers:} A handler is a function that processes messages and updates the module's state accordingly. To implement a handler, create a new Golang function in the module's handler.go file. This function should have the following signature:

\begin{lstlisting}[language=Go]
func NewHandler(k Keeper) sdk.Handler {
// Handler implementation
}
\end{lstlisting}

Within the handler, use a switch statement to process each message type and call the corresponding keeper methods to update the state. Also, emit relevant events to notify external systems about state changes. For example, a handler for the NFT module we defined in the previous section might look like this:

\begin{lstlisting}[language=Go]
package nft

func NewHandler(k Keeper) sdk.Handler {
return func(ctx sdk.Context, msg sdk.Msg) (*sdk.Result, error) {
switch msg := msg.(type) {
case MsgCreateNFT:
return handleMsgCreateNFT(ctx, k, msg)
case MsgTransferNFT:
return handleMsgTransferNFT(ctx, k, msg)
default:
return nil, sdkerrors.Wrap(sdkerrors.ErrUnknownRequest, "unrecognized NFT message type")
}
}
}
\end{lstlisting}

\textbf{Keepers:} Keepers are responsible for managing the module's state and providing an interface to interact with it. A keeper is implemented as a Golang struct, which is defined within the module's keeper.go file. This struct should contain a reference to the module's store and any other required dependencies.

To define a keeper, create a new Golang struct in the module's keeper folder and define methods that interact with the module's store. For example, a keeper for the NFT module might look like this:

\begin{lstlisting}[language=Go]
package keeper

type Keeper struct {
storeKey sdk.StoreKey
cdc *codec.Codec
}

func (k Keeper) CreateNFT(ctx sdk.Context, owner sdk.AccAddress, metadata string) (string, error) {
// Implementation for creating an NFT and updating the store
}

func (k Keeper) TransferNFT(ctx sdk.Context, sender sdk.AccAddress, recipient sdk.AccAddress, nftID string) error {
// Implementation for transferring an NFT and updating the store
}
\end{lstlisting}

Handlers and keepers are critical components of a custom module, providing the means to process messages and manage the module's state. By implementing these components, developers can define custom logic and create blockchain applications tailored to specific use cases.
%
\subsection{Module migration and versioning}

As blockchain applications evolve, it is crucial to maintain compatibility with previous versions and update the state and logic of modules. Module migration and versioning are essential aspects of the Cosmos SDK that allow developers to manage these updates. In this section, we will discuss how to handle module migrations and versioning in a Cosmos SDK application.

\textbf{Module versioning:} Each module in the Cosmos SDK has a version number that indicates its compatibility with the application state. When creating a new module, it is essential to define an initial version number in the module's types/version.go file. This version number should be incremented when breaking changes are introduced, ensuring that any migration processes are executed correctly.

\begin{lstlisting}[language=Go]
package types

const (
// ModuleName is the name of the module
ModuleName = "mymodule"

// StoreKey is the store key string for mymodule
StoreKey = ModuleName

// RouterKey is the message route for mymodule
RouterKey = ModuleName

// QuerierRoute is the querier route for mymodule
QuerierRoute = ModuleName

// DefaultParamspace is the default name for the parameter store
DefaultParamspace = ModuleName

// Version is the module's version number
Version = 1
)
\end{lstlisting}

\textbf{Module migration:} When a module's version is updated, it is necessary to implement a migration function that upgrades the module's state to be compatible with the new version. This migration function should be defined in the module's migration.go file and registered with the application's migration manager during the application initialization process.

The migration function should have the following signature:

\begin{lstlisting}[language=Go]
func Migrate(ctx sdk.Context, k Keeper, oldVersion uint64) error {
// Migration implementation
}
\end{lstlisting}

Within the migration function, implement the necessary logic to upgrade the module's state from the old version to the new version. For example, this may involve updating the data structure of stored items, removing deprecated items, or adding new items to the store.

When the application is upgraded, the migration manager will execute the registered migration functions for each module, ensuring that the application state is updated to be compatible with the new module versions.

Module migration and versioning are vital aspects of the Cosmos SDK that enable developers to manage updates to their blockchain applications. By properly implementing and registering migration functions, developers can ensure that their applications remain compatible with new module versions and avoid potential issues during the upgrade process.

\subsection{Integrating custom modules with a Cosmos SDK application}

After implementing a custom module with its types, messages, events, handlers, and keepers, the next step is to integrate it into your Cosmos SDK application. In this section, we will discuss the process of integrating custom modules with a Cosmos SDK application.

\textbf{1. Import the custom module:} First, import the custom module into your application's app.go file. Add the import statement at the beginning of the file, along with the other module imports.

\begin{lstlisting}[language=Go]
import (
// ...
"github.com/yourusername/yourapp/x/mymodule"
)
\end{lstlisting}

\textbf{2. Define the AppModule:} Create an instance of the AppModule for your custom module in the app.go file. The AppModule is responsible for managing the module's lifecycle within the application.

\begin{lstlisting}[language=Go]
mymoduleAppModule := mymodule.NewAppModule(appCodec, app.MymoduleKeeper)
\end{lstlisting}

\textbf{3. Register the AppModule:} Register the AppModule with the application's ModuleManager. This ensures that the module is initialized, started, and managed by the application.

\begin{lstlisting}[language=Go]
app.mm = module.NewManager(
// ...
mymoduleAppModule,
)
\end{lstlisting}

\textbf{4. Add the module's keys:} Add the module's key to the application's key manager in the app.go file. This allows the module to access its store within the application state.

\begin{lstlisting}[language=Go]
app.MymoduleKey = sdk.NewKVStoreKey(mymodule.StoreKey)
app.MemKeys[mymodule.ModuleName] = sdk.NewMemoryStoreKey(mymodule.ModuleName)
\end{lstlisting}

\textbf{5. Initialize the module's keeper:} Initialize the module's keeper in the app.go file. This step is essential to allow the module to interact with its store and manage its state.

\begin{lstlisting}[language=Go]
app.MymoduleKeeper = mymodule.NewKeeper(
appCodec, app.MymoduleKey, app.AccountKeeper, app.BankKeeper,
)
\end{lstlisting}

\textbf{6. Register the module's routes:} Register the module's message routes and querier routes in the app.go file. This step ensures that the application can handle the module's messages and queries.

\begin{lstlisting}[language=Go]
app.Router().AddRoute(mymodule.RouterKey, mymodule.NewHandler(app.MymoduleKeeper))
app.QueryRouter().AddRoute(mymodule.QuerierRoute, mymodule.NewQuerier(app.MymoduleKeeper))
\end{lstlisting}

By following these steps, you can successfully integrate a custom module into your Cosmos SDK application. Once the custom module is integrated, the application can manage the module's lifecycle, process its messages and queries, and interact with its state. This allows developers to extend the functionality of their blockchain applications by adding custom modules tailored to their specific requirements.

\section{Inter-Blockchain Communication (IBC)}

\subsection{Overview of IBC architecture and concepts}

The Inter-Blockchain Communication (IBC) protocol is a crucial component of the Cosmos ecosystem, enabling interoperability between different blockchains. In this section, we will discuss the core concepts and architecture of the IBC protocol, using examples from the Carbon-Cosmos project.

\textbf{1. Clients:} Clients are light clients of other chains within the IBC ecosystem. They track the consensus state of remote chains and verify the authenticity of data received from those chains. Clients help establish a secure communication channel between two chains.

\begin{lstlisting}[language=Go]
// From carboncredit/interchainnft/keeper/ibc.go
func (k Keeper) OnRecvPacket(ctx sdk.Context, packet channeltypes.Packet, data types.NFTPacketData) error {
// Validate the packet data and use the packet.ClientID to verify the data from the other chain
}
\end{lstlisting}

\textbf{2. Connections:} Connections are established between two chains by setting up a pair of light clients, one on each chain. A connection ensures that both chains can trust the data they receive from each other.

\begin{lstlisting}[language=Go]
// From carboncredit/interchainnft/keeper/ibc.go
func (k Keeper) OnAcknowledgementPacket(ctx sdk.Context, packet channeltypes.Packet, data types.NFTPacketData, ack channeltypes.Acknowledgement) error {
// Acknowledge the packet and process the data accordingly, relying on the secure connection
}
\end{lstlisting}

\textbf{3. Channels:} Channels are built on top of connections, providing a secure and ordered communication pathway between two chains. Channels handle data packets and can enforce ordering guarantees, ensuring that packets are processed in the order they were sent.

\begin{lstlisting}[language=Go]
// From carboncredit/interchainnft/keeper/ibc.go
func (k Keeper) SendNFTPacket(ctx sdk.Context, nftID string, srcPort, srcChannel string, timeoutHeight clienttypes.Height, timeoutTimestamp uint64, nft types.BaseNFT) error {
// Create a new NFTPacketData and send it through the specified channel
}
\end{lstlisting}

By understanding these IBC concepts and how they are implemented in the Carbon-Cosmos project, you can build applications that leverage the power of interoperability and communication between different blockchains in the Cosmos ecosystem.

\subsection{IBC modules, channels, and packets}

In this section, we will delve deeper into IBC modules, channels, and packets, using examples from the Carbon-Cosmos project. These components facilitate the transfer of data between different blockchains in the Cosmos ecosystem.

\textbf{1. IBC modules:} IBC modules are custom modules that implement the IBC protocol in a Cosmos SDK application. They handle the logic related to cross-chain communication, such as sending and receiving packets, and managing channels and connections.

In Carbon-Cosmos, the interchainnft module is an IBC module that allows the transfer of NFTs between different chains.

\begin{lstlisting}[language=Go]
// From carboncredit/interchainnft/types/msgs.go
type MsgSendNFT struct {
NFTID string json:"nft_id" yaml:"nft_id"
Sender sdk.AccAddress json:"sender" yaml:"sender"
Recipient sdk.AccAddress json:"recipient" yaml:"recipient"
SourcePort string json:"source_port" yaml:"source_port"
SourceChannel string json:"source_channel" yaml:"source_channel"
}
\end{lstlisting}

\textbf{2. Channels:} Channels provide a secure and ordered communication pathway between two chains. They are built on top of connections and are responsible for handling data packets. Channels can enforce ordering guarantees to ensure that packets are processed in the order they were sent.

\begin{lstlisting}[language=Go]
// From carboncredit/interchainnft/keeper/ibc.go
func (k Keeper) SendNFTPacket(ctx sdk.Context, nftID string, srcPort, srcChannel string, timeoutHeight clienttypes.Height, timeoutTimestamp uint64, nft types.BaseNFT) error {
// Create a new NFTPacketData and send it through the specified channel
}
\end{lstlisting}

\textbf{3. Packets:} Packets are the units of data sent between chains using IBC channels. They contain the actual data being transferred and metadata such as a sequence number, timeout height, and timeout timestamp. Packets can be acknowledged or timed out, allowing applications to handle different scenarios, such as refunds or retries.

\begin{lstlisting}[language=Go]
// From carboncredit/interchainnft/types/packet.go
type NFTPacketData struct {
NFTID string json:"nft_id" yaml:"nft_id"
TokenURI string json:"token_uri" yaml:"token_uri"
Sender string json:"sender" yaml:"sender"
Recipient string json:"recipient" yaml:"recipient"
}
\end{lstlisting}

By understanding and implementing IBC modules, channels, and packets in your Cosmos SDK application, you can enable seamless communication and data transfer between different blockchains, harnessing the full potential of the Cosmos ecosystem.


\subsection{Implementing IBC logic in a custom module}

Implementing IBC logic in a custom module involves defining the IBC packet data structure, creating messages for sending packets, and handling packets in the module's keeper. In this section, we will walk through these steps using the interchainnft module from the Carbon-Cosmos project as an example.

\textbf{1. Define IBC packet data structure:} First, define the IBC packet data structure for your custom module. This structure should contain the data you want to transfer between chains.

In the interchainnft module, the NFTPacketData structure is used to transfer NFTs.

\begin{lstlisting}[language=Go]
// From carboncredit/interchainnft/types/packet.go
type NFTPacketData struct {
NFTID string json:"nft_id" yaml:"nft_id"
TokenURI string json:"token_uri" yaml:"token_uri"
Sender string json:"sender" yaml:"sender"
Recipient string json:"recipient" yaml:"recipient"
}
\end{lstlisting}

\textbf{2. Create messages for sending packets:} Next, create messages that allow users to send IBC packets from your custom module. These messages should include the necessary information to initiate the packet transfer.

In the interchainnft module, the MsgSendNFT message is used for this purpose.

\begin{lstlisting}[language=Go]
// From carboncredit/interchainnft/types/msgs.go
type MsgSendNFT struct {
NFTID string json:"nft_id" yaml:"nft_id"
Sender sdk.AccAddress json:"sender" yaml:"sender"
Recipient sdk.AccAddress json:"recipient" yaml:"recipient"
SourcePort string json:"source_port" yaml:"source_port"
SourceChannel string json:"source_channel" yaml:"source_channel"
}
\end{lstlisting}

\textbf{3. Handle packets in the module's keeper:} Finally, implement the necessary functions in your custom module's keeper to handle sending and receiving packets, as well as acknowledgements and timeouts.

In the interchainnft module, the SendNFTPacket function sends NFT packets, while the OnRecvNFTPacket function handles received NFT packets.

\begin{lstlisting}[language=Go]
// From carboncredit/interchainnft/keeper/ibc.go
func (k Keeper) SendNFTPacket(ctx sdk.Context, nftID string, srcPort, srcChannel string, timeoutHeight clienttypes.Height, timeoutTimestamp uint64, nft types.BaseNFT) error {
// Create a new NFTPacketData and send it through the specified channel
}

func (k Keeper) OnRecvNFTPacket(ctx sdk.Context, packet channeltypes.Packet, data types.NFTPacketData) error {
// Process the received NFTPacketData and update the module's state
}
\end{lstlisting}

By following these steps, you can implement IBC logic in your custom Cosmos SDK module, enabling cross-chain communication and data transfer in your application.

\subsection{Practical examples: ICS-20 and ICS-721}

ICS-20 and ICS-721 are two practical examples of the Inter-Blockchain Communication (IBC) protocol in action. ICS-20 is a token transfer standard for fungible tokens, while ICS-721 is an extension of IBC for non-fungible tokens (NFTs). In this section, we will discuss how these standards are applied in the Carbon-Cosmos project.

\textbf{ICS-20:} The Carbon-Cosmos project does not implement ICS-20 directly, as it primarily deals with NFTs representing carbon credits. However, ICS-20 serves as a reference for creating custom IBC modules like the one in Carbon-Cosmos. You can find more information about ICS-20 in the Cosmos SDK's ibc-transfer module, which handles the transfer of fungible tokens across chains.

\textbf{ICS-721:} The interchainnft module in Carbon-Cosmos is an example of an ICS-721 implementation for NFTs. This module allows the transfer of NFTs between different chains in the Cosmos ecosystem, enabling cross-chain interactions for carbon credit NFTs.

Here are the main components of the interchainnft module that enable ICS-721 functionality:

\textbf{1. NFTPacketData:} This data structure defines the information to be transferred between chains as an IBC packet.

\begin{lstlisting}[language=Go]
// From carboncredit/interchainnft/types/packet.go
type NFTPacketData struct {
NFTID string json:"nft_id" yaml:"nft_id"
TokenURI string json:"token_uri" yaml:"token_uri"
Sender string json:"sender" yaml:"sender"
Recipient string json:"recipient" yaml:"recipient"
}
\end{lstlisting}

\textbf{2. MsgSendNFT:} This message type allows users to initiate the transfer of an NFT between chains.

\begin{lstlisting}[language=Go]
// From carboncredit/interchainnft/types/msgs.go
type MsgSendNFT struct {
NFTID string json:"nft_id" yaml:"nft_id"
Sender sdk.AccAddress json:"sender" yaml:"sender"
Recipient sdk.AccAddress json:"recipient" yaml:"recipient"
SourcePort string json:"source_port" yaml:"source_port"
SourceChannel string json:"source_channel" yaml:"source_channel"
}
\end{lstlisting}

\textbf{3. Keeper functions:} The interchainnft module's keeper includes functions to handle IBC packets, acknowledgements, and timeouts, such as SendNFTPacket and OnRecvNFTPacket.

\begin{lstlisting}[language=Go]
// From carboncredit/interchainnft/keeper/ibc.go
func (k Keeper) SendNFTPacket(ctx sdk.Context, nftID string, srcPort, srcChannel string, timeoutHeight clienttypes.Height, timeoutTimestamp uint64, nft types.BaseNFT) error {
// Create a new NFTPacketData and send it through the specified channel
}

func (k Keeper) OnRecvNFTPacket(ctx sdk.Context, packet channeltypes.Packet, data types.NFTPacketData) error {
// Process the received NFTPacketData and update the module's state
}
\end{lstlisting}

By implementing these components, the Carbon-Cosmos project demonstrates how to use IBC standards like ICS-721 to enable cross-chain functionality for custom applications.

\textbf{4. IBC client and CLI:} To interact with the IBC functionalities provided by the interchainnft module, Carbon-Cosmos includes a client and CLI commands. The tx.go and query.go files under carboncredit/interchainnft/client/cli contain the necessary code for sending and querying NFT transfers.

\begin{lstlisting}[language=Go]
// From carboncredit/interchainnft/client/cli/tx.go
func CmdSendNFT() *cobra.Command {
// Define a new command for sending NFTs via IBC
}

// From carboncredit/interchainnft/client/cli/query.go
func CmdQueryNFT() *cobra.Command {
// Define a new command for querying NFTs via IBC
}
\end{lstlisting}

These examples demonstrate how ICS-20 and ICS-721 standards can be applied to the Carbon-Cosmos project. By understanding the implementation of these standards, developers can create custom IBC modules for their own applications and enable seamless cross-chain interactions.




\section{Smart Contract Development in Cosmos Ecosystem}

\subsection{Introduction to CosmWasm and its comparison to Solidity}
\subsection{Writing smart contracts using Rust and CosmWasm}
\subsection{Deploying and interacting with CosmWasm smart contracts}
\subsection{Best practices for CosmWasm smart contract development}

\section{Testing and Simulation}

\subsection{Unit testing in Golang}
\subsection{Integration testing for Cosmos SDK applications}
\subsection{Simulation testing with the Cosmos SDK}
\subsection{Testing IBC functionality}

\section{Deployment and Network Management}

\subsection{Setting up a local testnet with Cosmos SDK}
\subsection{Deploying a Cosmos SDK application to a public testnet}
\subsection{Upgrading and governance in the Cosmos ecosystem}
\subsection{Performance and security considerations}

\section{Building Real-World Applications}

\subsection{Case studies of successful Cosmos-based projects}
\subsection{Design patterns and best practices for Cosmos SDK applications}
\subsection{Interoperability with Ethereum and other blockchain ecosystems}
\subsection{Future developments in the Cosmos ecosystem}


\title{Introduction to Golang: Basics of Golang Syntax}



\end{document}